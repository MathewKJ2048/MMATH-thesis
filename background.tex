%======================================================================
\chapter{Background}
%======================================================================


%----------------------------------------------------------------------
\section{Alloy}
%----------------------------------------------------------------------

%----------------------------------------------------------------------
\section{Dash}
%----------------------------------------------------------------------

%----------------------------------------------------------------------
\section{TLA+}
%----------------------------------------------------------------------


Objects whose value do not change with the traces of the model are called CONSTANTS in TLA+. Often, these CONSTANTS are given values in the .cfg file accompanying the .tla file fed into the TLC model checker. These objects correspond to instances generated by the Alloy Analyzer when a Dash+ model is run after translation to Alloy.

%----------------------------------------------------------------------
\section{Configuration generation}
%----------------------------------------------------------------------



In a paper \cite{blast} by Cunha et. Somson, the authors introduce a tool called Blast, which enables the use of TLA+ to find these objects during the model-checking phase, rather that having the user set specific values manually in the .cfg file. 


The .tla file and .cfg file are augmented with annotations read by blast. These annotations determine the type and scope of the configurations to be tested. Blast produces a modified .tla file and a modified .cfg file. 


Blast performs the following procedure on the input files:

\begin{enumerate}
	\item For every CONSTANT $C$ in the original .tla file, Blast adds a VARIABLE $V_{C}$ in the modified .tla file. (CONSTANTS with no type annotations are ignored by Blast)
	\item Blast makes these variables keep the same values during the trace, using the UNCHANGED modifier in TLA+ (equivalent to appending /\ $V_{C}' == V_{C}$ to every transition).
	\item Blast adds the type of the variable to the TypeOK relation in TLA+. Blast ignores CONSTANTS whose type is not specified in the annotation, which remain as CONSTANTS in the output files produced by Blast. 
	\item Blast constructs compound types (such as sets, relations, tuples, et. cetera.) from the annotation, which are constructed using basic types (such as Nat). Blast identifies basic types by the use of all caps identifiers.
	\item Blast modifies the Init relation by adding $V_{C} \in C$, where $C$ is now a CONSTANT.
	\item The value of $C$ in the .cfg file is a set whose size is determined by the scope annotations read by Blast.
\end{enumerate}



